\beginsong{Titel des Liedes}[
	index={<erste zeile>},
	wuw={Worte und Weise falls es die gleiche Person ist},
	jahr={jahr für worte und weise, alternativ j:},
	mel={Autor der Melodie. ALternativ melodie, weise},
	meljahr={jahr der Melodie. Alternativ meljahr, weisej, weisejahr},
	txt={Autor des Textes. Alternativ text, worte},
	txtjahr={Jahr des Textes. Alternativ txtjahr, textjahr, wortej, wortejahr},
	alb={Album: alternativ album},
	lager={Lager auf dem Das Lied geschrieben wurde},
	bo={Seite im Bock},
]


\beginverse
D\[a]as hier\[e] ist d\[F]ie erste Str\[C]ophe,
Man\[Esus4] schreibt \[H]die Akko\[G7]rde einfach
ü\[E]ber de\[d]n Text.\[H] 
\endverse


\beginverse*
\[G]Das h\[F]ier is\[E]t auc\[D]h ein\[C]e Strophe,
a\[G]lerdings o\[H]hne N\[c]ummer
\endverse*


\beginrefrain
Da\[Esus2]s ist der Refrain. Der funktioniert wie alle Strophen auch.
beachte: \[A]Akkorde werden l\[B]inksbündig eingefügt.
\endrefrain


\beginverse
Die Nummerierten strophen werden weitergezählt.
Das hier ist also die zweite Strophe. Hier stehen
keine Akkorde.
N\[C]ur in d\[D]er le\[E]tzten zeile stehen Akkorde
\endverse


\beginverse
D\[C#]ie Dritte S\[E5]trophe hat wieder Akk\[C]orde \[D]  \[Dsus2]      \[D]  \[e]  
war\[C]um auch immer. vielleicht ist die Melodie \[a]anders :)
\endverse


\beginverse*
dieser Vers hat keine Akkorde. Beachte, dass durch den leerraum eine 
ne\[D]ue Strophe \[E]entsteht. Hier gibt es \[f#]akkorde.
so schreibt man einen vers ohne Akkorde. 
\endverse*


\endsong